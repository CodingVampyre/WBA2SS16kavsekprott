\documentclass[10pt]{article}
\usepackage{ngerman}
\usepackage{graphicx}
\usepackage{a4wide}
\usepackage{ucs}
\usepackage[ansinew]{inputenc}
\usepackage{helvet}
\usepackage{float}
\usepackage[T1]{fontenc}
\usepackage{color}
\usepackage{pgfplots}
\renewcommand{\familydefault}{\sfdefault}
{\renewcommand{\arraystretch}{2}

\title{REST-Spezifikationen}
\author{Kavsek and Prott}

\begin{document}

\maketitle
\begin{table}[H]
\begin{center}
	\begin{tabular}{|c|c|c|c|c|}
		\hline
		Ressource & Methodik & Semantik & content-type (req) & content-type (res) \\
		\hline
		\hline
		/about & GET & Liefern einer Information & text/plain & text/plain \\
		/users/:name & GET & Liefern eines Users & text/plain & application/json \\
		/users/:name & POST & Einf�gen eines neuen Users & application/json & text/plain \\
		\hline
	\end{tabular}
\end{center}
\end{table}

\end{document}